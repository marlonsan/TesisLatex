\documentclass[10pt,a4paper]{Beamer}
\usepackage[spanish,es-tabla]{babel} % Idioma español con tablas
\usepackage{graphicx}
\usepackage{xcolor}
\usepackage{color}
\usepackage{alltt}
\usepackage{times}
\usepackage{amsmath}
\usepackage{amssymb}
\usepackage{amsfonts}
\usepackage[utf8]{inputenc}         % Para escribir en castellano
\usepackage[T1]{fontenc}
\usepackage{color}
\usepackage{alltt}
\usepackage{times}
\usepackage{latexsym}
\usepackage{anyfontsize}
\setcounter{secnumdepth}{3} %para que ponga 1.1.1.1 en subsubsecciones
\setcounter{tocdepth}{3}    % para que ponga subsubsecciones en el indice
\usepackage{setspace}       % Usado para %\onehalfspace  \doublespacing  \singlespace
\usepackage{booktabs}       % Para formar tablas

\usepackage{multirow}
\usepackage{multicol}
\usepackage{array,colortbl}

\usepackage[round]{natbib} 
\bibliographystyle{apalike}
 
\newcommand{\yy}{{\'{\i} }}
\newcommand{\y}{\'{\i}}
\textheight 21cm

\setlength{\fboxsep}{0pt}


\usetheme{Boadilla} 

\title[Nombre de la tesis]{Nombre de la tesis}
%\subtitle[Nombre de la tesis]{Nombre de la tesis}

\date[05/07/2017]{Defensa de tesis 05/07/2017}

\author[Edgar Peche Perlado, Manuel Perez Yon]
{
Edgar M. Peche Perlado\inst{}\\[-2mm]
\vskip 0.1cm
\texttt{\small edpe20@hotmail.com}\\
\vskip 0.2cm
    \and
Manuel A. Pérez Yon\inst{}\\[-2mm]
\vskip 0.1cm
\texttt{\small m.perezyon94@gmail.com}\\
}


\institute{Universidad Nacional de Trujillo \\
Facultad de Ciencias Físicas y Matemáticas\\
Escuela Académico Profesional de Informática}

\logo{\includegraphics[scale=.1]{LOGO_unt}}



\begin{document}
\frame{\titlepage}

\AtBeginSection[]{
\begin{frame}
\frametitle{
\begin{center}
{Nombre de la tesis}
\end{center}
}
\tableofcontents[currentsection]
\end{frame}
}

\frame{
\begin{abstract}
La investigación bibliográfica revela una preocupación de los gobiernos ... . 
\vskip 0.2cm 
En este sentido esta investigación tiene como objetivo principal modelar y planificar una red de logística reversa para una región urbana, dimensionando el flujo de RSU que será movido a lo largo de la red, el número y capacidad de las estaciones de colecta, de las unidades productivas y especiales necesarias para su colecta, transporte y disposición final. Los resultados muestran que es posible realizar un modelo matemático para este tipo de problemas, así como su aplicación en diversas regiones sin necesidad de grandes cambios en el modelo propuesto.    
\vskip 0.3cm
\hspace*{0.0cm}{\bf Palabras claves:} residuos sólidos urbanos, logística reversa, modelo matemático.
\end{abstract}
}


\section{Introducción}
\frame{
\begin{block}
{\Large{Introducción}}
\end{block}
\vskip 0.5cm
\begin{itemize}
\item<1-> A medida 
\vskip 0.5cm
\item<2-> Las   
\end{itemize}
}


\section{Motivación}
\frame{
\begin{block}
{\Large{Motivación}}
\end{block}
\vskip 0.5cm
\begin{itemize}
\item<1-> De esta forma,
\vskip 0.5cm
\item<2-> Por lo tanto, 
\end{itemize}
}

\section{Formulación del problema}
\frame{
\begin{block}
{\Large{Formulación del problema}}
\end{block}
\vskip 0.5cm
  En este trabajo, se propone discutir el modelo de red  de logística reversa basado en el problema del ruteo de vehículos para responder a la siguiente pregunta:
\vskip 1cm  
 \begin{center} 
     ?`Cómo viabilizar una red logística reversa en regiones urbanas minimizando los costos logísticos de ruteo y transporte de los RSU hasta su disposición final?
 \end{center}
}


\section{Importancia de la investigación} 
\frame{
\begin{block}
{\Large{Importancia de la investigación}}
\end{block}
\vskip 0.5cm
\begin{itemize}
\item<1-> La  
\vskip 0.3cm
\item<2-> El escenario ... en la tabla 1.1.
\end{itemize}
}




\frame{
\begin{table}[h!]
\centering
\caption{Estimativa en los programas de colecta selectiva formal (2008)}
\begin{tabular}{|c|c|c|c|}  \hline 
Residuos & Residuos colectados(t/día) & Residuos reciclados(t/año) \\ \hline 
Metales & 5 293 & 9 817 \\
Papeles & 23 997 & 3 827 \\ 
Plástico & 24 847 & 962 \\
Vidrio & 4 388 & 489  \\\hline
\end{tabular}
\begin{center}
{\small{Fuente: \cite{MMA}.}}
\end{center}
\end{table}
}


\section{Contribución de la investigación}
\frame{
\begin{block}
{\Large{Contribución de la investigación}}
\end{block}
\vskip 0.5cm
\begin{itemize}
\item<1-> La  
\vskip 0.3cm
\item<2-> El 
\end{itemize}
}

\section{Marco teórico}
\frame{
\begin{block}
{\Large{Marco teórico}}
\end{block}
\vskip 0.5cm
\begin{itemize}
\item<1-> {\bf Optimización combinatoria:}
\begin{itemize}
\item<2->
\vskip 0.3cm
\item<3->
\vskip 0.3cm
\item<4->
\vskip 0.3cm
\item<5->
\end{itemize}
\end{itemize}
}

\frame{
\begin{itemize}
\item<1-> {\bf Complejidad computacional:}
\begin{itemize}
\item<2->
\vskip 0.3cm
\item<3->
\vskip 0.3cm
\item<4->
\vskip 0.3cm
\item<5->
\end{itemize}
\end{itemize}
}

\frame{
\begin{itemize}
\item<1-> {\bf Metaheurísticas:}
\begin{itemize}
\item<2-> Algoritmos genéticos:
\vskip 0.3cm
\item<3-> Busca Tabú:
\vskip 0.3cm
\item<4-> Simulated annealing:
\vskip 0.3cm
\item<5-> Ant colony:
\end{itemize}
\end{itemize}
}



\section{Propuesta o tema central de la tesis}
\frame{
\begin{block}
{\Large{Propuesta o tema central de la tesis}}
\end{block}
\vskip 0.5cm
\begin{itemize}
\item<1-> 
\vskip 0.5cm
\item<2->
\vskip 0.5cm
\item<3->
\vskip 0.5cm
\item<4->
\vskip 0.5cm
\item<5->
\end{itemize}
}


\section{Resultados de la tesis}
\frame{
\begin{block}
{\Large{Resultados de la tesis}}
\end{block}
\vskip 0.5cm
Al culminar con la investigación se llegaron a resultados interesantes del punto de vista tanto teórico como computacional. ...
\begin{itemize}
\item<1-> {\bf Teóricos:}
\begin{itemize}
\item<2->
\vskip 0.3cm
\item<3->
\end{itemize}
\end{itemize}
}

\frame{
\begin{itemize}
\item<1-> {\bf Computacionales:}
\begin{itemize}
\item<2->
\vskip 0.3cm
\item<3->
\end{itemize}
\end{itemize}
}


\section{Consideraciones finales}
\frame{
\begin{block}
{\Large{Consideraciones finales}}
\end{block}
\vskip 0.5cm
\begin{itemize}
\item<1-> {\bf Conclusiones:}
\begin{itemize}
\item<2->
\vskip 0.3cm
\item<3->
\end{itemize}
\end{itemize}
}

\frame{
\begin{itemize}
\item<1-> {\bf Trabajos futuros:}
\begin{itemize}
\item<2->
\vskip 0.3cm
\item<3->
\end{itemize}
\end{itemize}
}

\section{Referencias bibliograficas}
\frame{
\begin{block}
{\Large{Referencias bibliograficas}}
\end{block}
\vskip 0.5cm
 \bibliography{Bibliografia} % Bibliografia formato APA
}
\end{document}
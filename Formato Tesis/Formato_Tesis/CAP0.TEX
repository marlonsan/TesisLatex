\documentclass[a4paper, 12pt]{report}

\usepackage[left = 3cm, top = 2.5cm, bottom = 1.5cm, right = 2cm]{geometry}
\usepackage{graphicx}
\usepackage[spanish,es-tabla]{babel} % Idioma español con tablas

\usepackage{amsmath}
\usepackage{amssymb}
\usepackage{amsfonts}
\usepackage[utf8]{inputenc}         % Para escribir en castellano
\usepackage[T1]{fontenc}
\usepackage{color}
\usepackage{alltt}
\usepackage{times}
\usepackage{latexsym}
\usepackage{anyfontsize}
\setcounter{secnumdepth}{3} %para que ponga 1.1.1.1 en subsubsecciones
\setcounter{tocdepth}{3}    % para que ponga subsubsecciones en el indice
\usepackage{setspace}       % Usado para %\onehalfspace  \doublespacing  \singlespace
\usepackage{booktabs}       % Para formar tablas
%\usepackage{longtable}     % Usado para diseñar grandes tablas.

%\usepackage{algorithmic}
%\usepackage{algorithm}


%%%%%%%%%%%%%%%%%%%%%%%%%%%%%%%
%%% Tables-related packages %%%
%%%%%%%%%%%%%%%%%%%%%%%%%%%%%%%
\usepackage{multirow}
\usepackage{multicol}
\usepackage{array,colortbl}

\usepackage[round]{natbib} 
\bibliographystyle{apalike}
 
\newcommand{\yy}{{\'{\i} }}
\newcommand{\y}{\'{\i}}
\textheight 21cm

\begin{document}

\baselineskip 1cm
\pagestyle{plain}
\input{cap1.tex}    % caratula del trabajo
\listoffigures      % indice de figuras
\addcontentsline{toc}{chapter}{Índice de Figuras}
\listoftables       % indice de tablas
\addcontentsline{toc}{chapter}{Índice de Tablas}
\tableofcontents    % indice de materias
\input{cap2.tex}    % capitulo de introduccion
\input{cap3.tex}    % capitulo de marco teorico
\input{cap4.tex}    % capitulo de cuerpo del trabajo
\input{cap5.tex}    % capitulo de Resultados
\input{cap6.tex}    % capitulo de consideraciones finales
 \bibliography{Bibliografia} % Bibliografia formato APA
\input{cap7.tex}    % Apendice
\end{document}






